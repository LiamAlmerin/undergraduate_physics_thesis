\section{Background and Motivation}
The bacterial flagellum represents one of the largest bacterial surface structures and enables bacteria to swim through liquids or swarm over semisolid surfaces.~\cite{Guttenplan2013}  Bacterial flagella are highly conserved molecular machines that have been extensively studied from the standpoint of their assembly, function and gene regulation. Understanding the biophysical mechanisms underlying the emergent patterning of these transenvelope nanomachines is a fundamental aspect of bacterial self-organization, with importance consequences for the efficacy of bacterial motility and resulting fitness. This thesis investigates how bacteria reproducibly establish spatial localization and number of flagella during each round of cell division. 

Recent experimental evidence has revealed the importance of intracellular proteins FlhF/FlhG in flagellar patterning.  It has been established that FlhF is a GTPase, like the cell division protein FtsZ whose polymerization is required for septation, and that FlhG is an ATPase, like the FtsZ-antagonizing protein MinD.  MinD homologues have been implicated not only in cell division, but are also involved in partitioning plasmids and flagellar-regulating chemotaxis clusters. In the cases mentioned above, the ATPase governs the spacing between the cargo. This suggests that FlhG may set the length scale between clusters of FlhF, which in turn instruct the cell where to insert new flagellar basal bodies into the membrane. Biochemically, FlhG binds to FlhF, enhancing FlhF GTP hydrolysis and interfering with its membrane localization. However, how this FlhG antagonism of FlhF controls flagellar patterning is unknown.

Given the homology between FlhG and the MinD ATPase, a starting point for a reaction-diffusion model of  FlhF/FlhG dynamics is the well-established MinCDE model governing correct positioning of the divisome at midcell. While regulation of  flagellation patterns  by FlhF/FlhG has been demonstrated, much information and key steps are as yet unknown; as such, mathematical  modeling and simulation can complement experimental approaches, choice of mutants, etc.  Our hypothesis is that the same biophysical mechanisms leading to regular,  “crystalline” or grid-like patterning in some cases generate polar localization of flagella in others, depending on the selected wavelength of the underlying patterns.

More generally, spatiotemporal oscillations of the Min proteins are a striking example of biological pattern formation. In a seminal 1952 paper, Alan Turing showed that symmetry breaking periodic patterns, or chemical ``morphogenesis", could arise as a result of the linear instability of a spatially uniform steady state in a reaction–diffusion system with a rapidly diffusing inhibitor and a slowly diffusing activator.~\cite{Turing1952}  Similarly, in most models of the Min system, it can be shown that a dynamic instability leads to self-organized patterning.~\cite {Kondo2018} Do patterns of FlhF/G localization, which in turn recruit or inhibit flagellar assembly, emerge spontaneously, via a Turing-like mechanism?  If so, what biologically relevant parameters in the underlying reaction-diffusion model governing the biochemical interactions of these proteins, the membrane and flagellar basal bodies, determine the wavelength of the instability and hence the length scale of flagellar patterns?  Can this wavelength be tuned, for example by varying the expression levels of the key protein(s), to achieve polar localization of flagella as observed in some bacterial species versus regular, uniformly distributed organization as observed in others?  How do the various underlying rates (rate at which new basal bodies are inserted into the membrane, rate of diffusion in the membrane, rate of cell division) give rise to flagellar number homeostasis? 



