\chapter{Conclusion and Future Work}

The coefficient of variation of the densities decreased to a point and then flattened out as the inputted insert rate increased. The effective insertion rate also appears to flatten out at the same values where the coefficient of variation flattens out. Through this, we can say that this data shows that the regularity of the basal body locations increases as the amount of basal bodies increase. However, only a single simulation was done per data point, so more would need to be done to say more about the relationship between regularity, inhibition zone size, density, and insertion rate. With more simulations, we would be able to calculate nearest neighbor, Clark-Evans, and distance from midcell distributions to compare with those found by~\cite{Guttenplan2013}.

Alongside additional simulations, this will be later extended to include coupling to FlhF/G dynamics. Another thing that would be possible would be to alter the inhibition zone to being circular instead of square. Firstly though, the insertion of new cell membrane material in the simulation will need to be changed to be added to center instead of randomly to be more physically correct.

Before running additional simulations, work will need to be done to better optimize the code. The computation of the growth and division of the cell membrane can be done before any computation of the basal body dynamics, since the size of the cell does not depend on the them. This will allow us to know the max amount of basal bodies in the cell, which will let us be able to create the vector holding the data about the basal bodies to be of fixed size, which will save resources from needing to resize the vector each time a new basal body is inserted. Additionally, we will alter particle insertion code to not recreate the matrix of allowed positions per particle insertion, but rather change the values in the matrix during the random walking or after all the movements have occurred during that iteration.